\documentclass[11pt]{article}

\usepackage{listings}
\usepackage{color}
%\usepackage[parfill]{parskip}
\usepackage{listings}
\usepackage{url}
\definecolor{mygreen}{rgb}{0,0.4,0}
\lstset{ %
frame=single,           % adds a frame around the code
basicstyle=\footnotesize\ttfamily,       % the size of the fonts that are used for the code
backgroundcolor=\color{white},  % choose the background color. You must add \usepackage{color}
commentstyle=\color{mygreen},
language=Bash,
stringstyle=\color{blue},
keywordstyle=\color{red},
showspaces=false,               % show spaces adding particular underscores
showstringspaces=false,         % underline spaces within strings
showtabs=false,                 % show tabs within strings adding particular underscores
tabsize=2,          % sets default tabsize to 2 spaces
captionpos=b,           % sets the caption-position to bottom
breaklines=true,        % sets automatic line breaking
breakatwhitespace=false,    % sets if automatic breaks should only happen at whitespace
escapeinside={\%*}{*)}          % if you want to add a comment within your code
}

\title{\textbf{AdNebulae}\\Cloud Management System}
\author{Josh Chase}

\begin{document}
\maketitle
\pagebreak

\newcommand\aN{\textit{AdNebulae} }
\newcommand\GOPATH{\$GOPATH}
\newcommand\PATH{\$PATH}
\newcommand\HOME{\$HOME}

\begin{abstract}
    Virtual machine provisioning and configuration have historically
    been separate, unrelated processes. Tools and services such as
    Openstack, Eucalyptus, and Amazon EC2 serve to abstract away
    hardware for the user while tools like Chef, Puppet, Ansible, and
    Salt abstract away the operating system. This separation of
    processes can, at best, pose a minor annoyance to administrators who
    are required to perform additional steps to fully provision a
    virtual machine. At worst, they can lead to inconsistencies and
    errors which can create outages at inopportune times. \aN was
    created to unify the two processes and provide a single utility
    responsible for virtual machine creation and configuration by
    integrating the Openstack and Chef APIs
\end{abstract}

\begin{tableofcontents}
\end{tableofcontents}

\pagebreak

\section{Installation}

\aN is written in Go, and, as such, only requires that the Go compiler
and git be installed in order to build it from source. This can be
accomplished in one command. Ensure that the \texttt{\GOPATH}
environment variable is set and run the following:

\begin{lstlisting}
# Set necessary environment variables
$ export GOPATH=$HOME/Go
$ export PATH=$GOPATH/bin:$PATH
# Install the AdNebulae CLI
$ go get github.com/Pursuit92/adnebulae/an-cli
\end{lstlisting}

This will install the \texttt{an-cli} binary to
\texttt{\GOPATH/bin/an-cli}. As long as \texttt{\GOPATH/bin} is in your
\texttt{\PATH} variable, you'll be able to run the command-line
interface to \aN like any other command.

\begin{lstlisting}
$ an-cli helpfull
Adnebulae CLI: Manage Openstack instances along with their configurations

Commands:
  create                Create a new VM
    floating-ip         Create a new floating ip
  delete                Delete a VM
    floating-ip         Delete a floating IP
  edit                  Edit Chef data
    databag             Edit databag
    node                Edit node
    role                Edit role
  floating-ip           Floating IP Management
    add                 Add a floating IP to a server
    remove              Remove a floating IP from a server
  helpfull              Show all commands
  list                  List VMs
    cookbooks           List available Cookbooks
    databags            List databags
    environments        List available environments
    flavors             List available Flavors
    floating-ip-pools   List floating IP poolss
    floating-ips        List floating IPs
    images              List available Images
    nets                List available Networks
  ...
\end{lstlisting}

\section{Configuration}

\aN is configured solely using environment variables, so the actual
configuration will be platform-dependent. The following variables must
be set for \aN to run:

\begin{description}
\item[\texttt{OS\_TENANT\_NAME}] Name of the Openstack Tenant to use for authentication
\item[\texttt{OS\_USERNAME}] Openstack Username
\item[\texttt{OS\_PASSWORD}] Openstack Password
\item[\texttt{CHEF\_ENDPOINT}] URL of the Chef server
\item[\texttt{CHEF\_VALIDATOR}] Path to the Chef Validator private key
\item[\texttt{CHEF\_USERNAME}] Chef Username
\item[\texttt{CHEF\_KEY\_FILE}] Path to the private key for Chef authentication
\end{description}

On a UNIX or Linux system, these variables would usually be set from the
\texttt{\HOME/.bashrc} file.

\begin{lstlisting}
export OS_TENANT_NAME=demo
export OS_USERNAME=demo
export OS_PASSWORD=demo
export CHEF_ENDPOINT=https://216.249.138.190:443
export CHEF_VALIDATOR=/home/josh/chef/.chef/chef-validator.pem
export CHEF_USERNAME=josh
export CHEF_KEY_FILE=/home/josh/chef/.chef/josh.pem
\end{lstlisting}

\pagebreak

\section{Usage}

\aN uses a simple command language to orchestrate the creation and
configuration of VMs. Most of its subcommands have multiple subcommands
themselves. With any partially complete command, the \texttt{-h} flag
can be passed to return more information about how it can be completed.

\begin{lstlisting}
$ an-cli -h
flag: help requested

Usage: an-cli [OPTIONS] [COMMAND] [arg...]

Adnebulae CLI: Manage Openstack instances along with their configurations

Commands:
  create        Create a new VM
  delete        Delete a VM
  edit          Edit Chef data
  floating-ip   Floating IP Management
  helpfull      Show all commands
  list          List VMs
  show          Show more info
  update        Update a VM

$ an-cli list -h
flag: help requested

Usage: an-cli list [OPTIONS] [COMMAND] [arg...]

List VMs

Commands:
  cookbooks           List available Cookbooks
  databags            List databags
  environments        List available environments
  flavors             List available Flavors
  floating-ip-pools   List floating IP poolss
  floating-ips        List floating IPs
  images              List available Images
  nets                List available Networks
  roles               List available Roles
  vms                 List VMs

\end{lstlisting}

\subsection{List}

The \texttt{list} subcommand is used for, you guessed it, listing
various items in either the Openstack environment or the Chef server. By
default, it lists the VMs that belong to the current Openstack Tenant.

\begin{lstlisting}
# List Openstack VMs
$ an-cli list vms
+----+------+--------+--------------------+-----------------...
| Id | Name | Status | Networks           | Run List        ...
+----+------+--------+--------------------+-----------------...
| ...| demo | ACTIVE | demo=10.0.0.2,21...| role[BASE],recip...
+----+------+--------+--------------------+-----------------...


# List Chef Cokbooks
$ an-cli list cookbooks
+---------------------+-------------------+
| Name                | Versions          |
+---------------------+-------------------+
| yum-mysql-community | 0.1.10            |
| postfix-dovecot     | 0.2.0             |
| yum-epel            | 0.4.0             |
| mysql               | 5.4.3             |
|                     | 5.4.0             |
| dovecot             | 1.0.0             |
| cron                | 1.4.0             |
| iptables            | 0.13.2            |
| postfixadmin        | 0.2.0             |
| chef-sugar          | 2.2.0             |
| bind                | 0.3.1             |
|                     | 0.3.0             |
| postfix             | 3.5.0             |
| yum                 | 3.2.4             |
  ...
\end{lstlisting}

As you can see from the VM listing, \aN attempts to unify information from both
Chef and Openstack wherever possible in order to provide admins with the most
complete picture possible.

\subsection{Show}

The \texttt{show} command is used to glean more information about a VM
or Chef item. Like with \texttt{list}, it defaults to showing VM info.
\pagebreak
\begin{lstlisting}
# Get information about a VM
$ an-cli show demo
+-------------+--------------------------------------+
| Property    | Value                                |
+-------------+--------------------------------------+
| Name        | demo                                 |
| Network     | demo=10.0.0.2,216.249.138.214        |
| Flavor      | m1.small                             |
| Image       | CentOS-6.5                           |
| Created     | 2014-12-08T02:13:59Z                 |
| Updated     | 2014-12-08T02:14:25Z                 |
| Status      | ACTIVE                               |
| Id          | 3f86df40-3b6b-4c8d-a952-b5e07b94b302 |
| TenantId    | f4709df496f440db9aa587f9568f09f3     |
| Environment | _default                             |
| RunList     | role[BASE],                          |
|             | recipe[bind]                         |
| Platform    | centos 6.5                           |
| Attributes  | {                                    |
|             |   "tags": []                         |
|             | }                                    |
+-------------+--------------------------------------+

# Show the contents of a Chef Databag
$ an-cli show databag zones jec_pw
+-------------------------------------------+
| Data                                      |
+-------------------------------------------+
| {                                         |
|   "conf": {                               |
|     "type": "master"                      |
|   },                                      |
|   "data": {                               |
|     "aliases": {                          |
|       "joshchase-frankfort.no-ip.org.": [ |
|         "fort"                            |
|       ]                                   |
|     },                                    |
|     "contact": "admin.jec.pw",            |
|     "expire": "2w",                       |
|     "generate": {},                       |
|     "hosts": {                            |
|       "mail": "216.249.138.192",          |
|       "ns1": "216.249.138.191",           |
|       "ns2": "216.249.138.191"            |
  ...
\end{lstlisting}

\subsection{Create}

The \texttt{create} command is where most of the \aN magic happens. It
leverages both the Openstack and Chef APIs to start up a VM and enroll it in
Chef in one fell swoop. Unlike the usual method of Chef enrollment, this
process requires no external access to the VM. This way, a VM on an
cloud-only private network, such as those given to each Openstack
Tenant, can easily be managed by Chef.

All but the desired name for the new VM are supplied using command-line
flags. Of the flags, all but \texttt{key-name} and \texttt{runlist} are
required.

\begin{lstlisting}
# VM creation options
$ an-cli create -h
flag: help requested

Usage: an-cli create [OPTIONS] [COMMAND] [arg...]

Create a new VM

Options:
  -chef=true: Enroll VM in chef
  -flavor="": New VM size
  -image="": Image to boot
  -key-name="": Keypair to use for the new instance
  -net="": Network for the new VM
  -runlist="": Chef run-list

# Create a demonstration VM
$ an-cli create -net demo \
                -flavor m1.small \
                -image CentOS-6.5 \
                -key-name josh-laptop \
                -runlist role[BASE] \
                test
+----------+--------------------------------------+
| Property | Value                                |
+----------+--------------------------------------+
| Name     | test                                 |
| Network  |                                      |
| Flavor   | m1.small                             |
| Image    | CentOS-6.5                           |
| Created  | 2014-12-09T02:31:28Z                 |
| Updated  | 2014-12-09T02:31:29Z                 |
| Status   | BUILD                                |
| Id       | 057807f9-15d3-41ea-b2de-1510dfe4e56c |
| TenantId | f4709df496f440db9aa587f9568f09f3     |
+----------+--------------------------------------+
\end{lstlisting}

The \texttt{create} command can also be used to allocate new
floating-IPs in Openstack. It only requires the name of the pool from
which to allocate the IP.

\begin{lstlisting}
# Allocate an IP from the "public" pool
$ an-cli create floating-ip public
+-------------+--------------------------------------+
| Property    | Value                                |
+-------------+--------------------------------------+
| Fixed IP    |                                      |
| ID          | 494da7df-6a01-488f-96bd-36a7fc99458f |
| Instance ID |                                      |
| IP          | 216.249.138.215                      |
| Pool        | public                               |
+-------------+--------------------------------------+
\end{lstlisting}

\subsection{Delete}

The \texttt{delete} subcommand is used for deleting VMs and
floating-IPs. In the process of deleting VMs, it also cleans up the
associated Chef nodes and clients if they exist.

\begin{lstlisting}
# Delete the VM named "test"
$ an-cli delete test

# Delete the floating-IP 216.249.138.215
$ an-cli delete floating-ip 216.249.138.215
\end{lstlisting}

\subsection{Update}

The \texttt{update} command is used to modify the Chef runlist or
environment for a VM. It takes the new runlist and/or environment as
flag arguments and the name or ID of the VM as a positional argument.

\begin{lstlisting}
$ an-cli update -runlist role[BASE],recipe[simplemail] demo
\end{lstlisting}

\subsection{Edit}

The \texttt{edit} command is used to do complex editing of Chef objects
- all of which are presented in JSON structures. Rather than modify them
with a single command, the \texttt{edit} subcommand will open the data
in whatever editor the \$EDITOR environment variable contains. If not
set, the editor defaults to \texttt{vim}. If the data is modified and
saved, it will be validated and uploaded to the Chef server after the
editor exits.

\begin{lstlisting}
# Edit the configuration for the VM named "demo"
$ an-cli edit node demo
Creating new file /tmp/fb227a13-b843-4ab6-aa1a-a7a9e5889428.json
File unmodified.
Done editing demo!

# Edit the Chef Databag item "jec_pw" in the "zones" bag
$ an-cli edit databag zones jec_pw
Creating new file /tmp/zones-jec_pw.json
File unmodified.
Done editing zones jec_pw!
\end{lstlisting}

\subsection{Floating-IP}

Finally, the \texttt{floating-ip} subcommand allows allocated
floating-ips to be added and removed from VMs to allow access to them
from outside of their private networks. Both subcommands have identical
arguments: the name of the server and the IP to be added/removed.

\begin{lstlisting}
# Add 216.249.138.214 to "demo"
$ an-cli floating-ip add demo 216.249.138.214

# Remove 216.249.138.214 from "demo"
$ an-cli floating-ip remove demo 216.249.138.214
\end{lstlisting}

\section{Source and Attributions}

The complete source of \aN can be found on GitHub under my account,
Pursuit92, along with all of its supporting libraries.

All of the supporting libraries except for the Chef API bindings (the
credit for which goes to Jesse Nelson $<$spheromak@gmail.com$>$) and the Go
standard library were written by me, Josh Chase
$<$jcjoshuachase@gmail.com$>$ and are unlicensed unless otherwise stated.

\end{document}
